\begin{exercise}
Show that \((\mathbf{Z} / m \mathbf{z}) \otimes_{\mathbf{Z}} (\mathbf{Z} / n \mathbf{Z}) = 0\) if \(m, n\) are coprime.
\end{exercise}

\begin{solution}
Since \(m\) and \(n\) are coprime, \(m\) is a unit in \(\mathbb{Z} / n \mathbb{Z}\).
Moreover, \((\mathbf{Z} / m \mathbf{Z}) \otimes_{\mathbf{Z}} (\mathbf{Z} / n \mathbf{Z})\) is generated by elements of the form \(x \otimes y\), where \(x \in \mathbb{Z} / m \mathbf{Z}\) and \(y \in \mathbb{Z} / n \mathbf{Z}\), and \(x \otimes y = x \otimes m m^{-1} y = m x \otimes m^{-1} y = 0 \otimes m^{-1} y = 0\).
Therefore, \((\mathbf{Z} / m \mathbf{z}) \otimes_{\mathbf{Z}} (\mathbf{Z} / n \mathbf{Z}) = 0\).
\end{solution}

In fact, here is a more general result that implies \exref{2}{1}.

\begin{claim}
Let \(\mathfrak{a}\) and \(\mathfrak{b}\) be ideals of a ring \(A\).
Then
\begin{equation*}
(A / \mathfrak{a}) \otimes_A (A / \mathfrak{b}) \cong A / (\mathfrak{a} + \mathfrak{b})
\end{equation*}
as \(A\)-algebras.
\end{claim}

\begin{proof}
Suppose \(a_1, a_2, b_1, b_2 \in A\) such that \(a_1 - a_2 \in \mathfrak{a}\) and \(b_1 - b_2 \in \mathfrak{b}\).
Then
\begin{equation*}
a_1 b_1 - a_2 b_2
= (a_1 - a_2) b_1 + a_2 (b_1 - b_2),
\end{equation*}
so \(a_1 b_1 - a_2 b_2 \in \mathfrak{a} + \mathfrak{b}\).
It follows that there is a well-defined function
\begin{align*}
f &: (A / \mathfrak{a}) \times (A / \mathfrak{b}) \to A / (\mathfrak{a} + \mathfrak{b}), &
f(x + \mathfrak{a}, y + \mathfrak{b}) = x y + \mathfrak{a} + \mathfrak{b}.
\end{align*}
Moreover, \(f\) is \(A\)-bilinear, so by the universal property of the tensor product there is a unique \(A\)-linear map \(\tilde{f} : (A / \mathfrak{a}) \otimes_A (A / \mathfrak{b}) \to A / (\mathfrak{a} + \mathfrak{b})\) such that \(\tilde{f}(a \otimes b) = f(a, b)\) for all \(a \in A / \mathfrak{a}\) and \(b \in A / \mathfrak{b}\).
Note that not only is \(\tilde{f}\) \(A\)-linear, it is in fact an \(A\)-algebra homomorphism.

Next, define the \(A\)-algebra homomorphism \(g : A \to (A / \mathfrak{a}) \otimes_A (A / \mathfrak{b})\) by
\begin{equation*}
g(x)
= (x + \mathfrak{a}) \otimes (1 + \mathfrak{b})
= (1 + \mathfrak{a}) \otimes (x + \mathfrak{b})
\end{equation*}
for \(x \in A\).
Let \(h_1 : A / \Ker(g) \to (A / \mathfrak{a}) \otimes_A (A / \mathfrak{b})\) denote the \(A\)-algebra homomorphism induced by \(g\): \(h(x + \Ker(g)) = g(x)\) for all \(x \in A\).
If \(x \in \mathfrak{a}\) and \(y \in \mathfrak{b}\), then
\begin{equation*}
g(x + y) = (x + \mathfrak{a}) \otimes (1 + \mathfrak{b}) + (1 + \mathfrak{a}) \otimes (b + \mathfrak{b}) = 0,
\end{equation*}
so \(\mathfrak{a} + \mathfrak{b} \subseteq \Ker(g)\).
It follows that there is an \(A\)-algebra homomorphism \(h_2 : A / (\mathfrak{a} + \mathfrak{b}) \to A / \Ker(g)\) given by \(h_2(x + \mathfrak{a} + \mathfrak{b}) = x + \Ker(g)\) for all \(a \in A\).
Now let
\begin{equation*}
\tilde{g} \coloneqq h_1 \circ h_2 : A / (\mathfrak{a} + \mathfrak{b}) \to (A / \mathfrak{a}) \otimes_A (A / \mathfrak{b}).
\end{equation*}
Then \(\tilde{g}\) is an \(A\)-algebra homomorphism, and calculation shows that it is the inverse of \(\tilde{f}\).
The claim follows.
\end{proof}