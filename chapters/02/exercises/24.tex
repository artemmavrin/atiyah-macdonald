\section{Flatness and Tor}

\begin{exercise}
If \(M\) is an \(A\)-module, the following are equivalent:
\begin{parts}
\part
\(M\) is flat;
\part
\(\Tor_n^A(M, N) = 0\) for all \(n > 0\) and all \(A\)-modules \(N\);
\part
\(\Tor_1^A(M, N) = 0\) for all \(A\)-modules \(N\).
\end{parts}
[To show that \partref{2}{24}{i} \(\implies\) \partref{2}{24}{ii}, take a free resolution of \(N\) and tensor it with \(M\).
Since \(M\) is flat, the resulting sequence is exact and therefore its homology groups, which are the \(\Tor_n^A(M, N)\), are zero for \(n > 0\).
To show that \partref{2}{24}{iii} \(\implies\) \partref{2}{24}{i}, let \(0 \to N^\prime \to N \to N^{\prime\prime} \to 0\) be an exact sequence.
Then, from the \(\Tor\) sequence,
\begin{equation*}
\Tor_1(M, N) \to M \otimes N^\prime \to M \otimes N \to M \otimes N^{\prime\prime} \to 0
\end{equation*}
is exact.
Since \(\Tor_1(M, N^{\prime\prime}) = 0\) it follows that \(M\) is flat.]
\end{exercise}

\begin{solution}
Suppose \(M\) is flat, and let \(N\) be an \(A\)-module.
Consider a free resolution \(\cdots \to F_2 \to F_1 \to F_0 \to N \to 0\).
Tensoring this exact sequence with \(M\) yields a chain complex
\begin{equation*}
\cdots \to F_2 \otimes_A M \to F_1 \otimes_A M \to F_0 \otimes_A M \to 0
\end{equation*}
which is exact at each positive index since \(M\) if flat.
Thus the homology groups of this chain complex, which are the \(\Tor_n^A(M, N)\), are all zero.
Thus \partref{2}{24}{i} implies \partref{2}{24}{ii}.

Moreover, \partref{2}{24}{iii} follows immediately from \partref{2}{24}{ii}.

Finally, suppose \(\Tor_1^A(M, N) = 0\) for all \(A\)-modules \(N\).
Consider a short exact sequence \(0 \to N^\prime \to N \to N^{\prime\prime} \to 0\) of \(A\)-modules.
Then there is a canonical long exact sequence
\begin{equation*}
\begin{tikzcd}[row sep=small, column sep=small]
\cdots \arrow[r] & \Tor_2^A(M, N^\prime) \arrow[r] & \Tor_2^A(M, N) \arrow[r] \arrow[d, phantom, ""{coordinate, name=Z2}] & \Tor_2^A(M, N^{\prime\prime})
\arrow[lld,
    rounded corners,
    to path={ -- ([xshift=2ex]\tikztostart.east)
    |- (Z2) [near end]\tikztonodes
    -| ([xshift=-2ex]\tikztotarget.west)
    -- (\tikztotarget)}]
\\
& \Tor_1^A(M, N^\prime) \arrow[r] & \Tor_1^A(M, N) \arrow[r] \arrow[d, phantom, ""{coordinate, name=Z1}] & \Tor_1^A(M, N^{\prime\prime})
\arrow[lld,
    rounded corners,
    to path={ -- ([xshift=2ex]\tikztostart.east)
    |- (Z1) [near end]\tikztonodes
    -| ([xshift=-2ex]\tikztotarget.west)
    -- (\tikztotarget)}]
\\
&
M \otimes_A N^\prime \arrow[r] & M \otimes_A N \arrow[r] & M \otimes_A N^{\prime\prime} \arrow[r] & 0.
\end{tikzcd}
\end{equation*}
Since \(\Tor_1^A(M, N^{\prime\prime}) = 0\), the end of this long exact sequence is the short exact sequence
\begin{equation*}
0 \to M \otimes_A N^\prime \to M \otimes_A N \to M \otimes_A N^{\prime\prime} \to 0.
\end{equation*}
Thus, \(M\) is flat, so \partref{2}{24}{iii} implies \partref{2}{24}{i}.
\end{solution}