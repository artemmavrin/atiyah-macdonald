\section{Flatness and Tor}

\begin{exercise}
If \(M\) is an \(A\)-module, the following are equivalent:
\begin{parts}
\part
\(M\) is flat;
\part
\(\Tor_n^A(M, N) = 0\) for all \(n > 0\) and all \(A\)-modules \(N\);
\part
\(\Tor_1^A(M, N) = 0\) for all \(A\)-modules \(N\).
\end{parts}
[To show that \partref{2}{24}{i} \(\implies\) \partref{2}{24}{ii}, take a free resolution of \(N\) and tensor it with \(M\).
Since \(M\) is flat, the resulting sequence is exact and therefore its homology groups, which are the \(\Tor_n^A(M, N)\), are zero for \(n > 0\).
To show that \partref{2}{24}{iii} \(\implies\) \partref{2}{24}{i}, let \(0 \to N^\prime \to N \to N^{\prime\prime} \to 0\) be an exact sequence.
Then, from the \(\Tor\) sequence,
\begin{equation*}
\Tor_1(M, N) \to M \otimes N^\prime \to M \otimes N \to M \otimes N^{\prime\prime} \to 0
\end{equation*}
is exact.
Since \(\Tor_1(M, N^{\prime\prime}) = 0\) it follows that \(M\) is flat.]
\end{exercise}