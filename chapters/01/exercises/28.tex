\begin{exercise}
Let \(f_1, \ldots, f_m\) be elements of \(k[t_1, \ldots, t_n]\).
They determine a \emph{polynomial mapping} \(\phi : k^n \to k^m\):
if \(x \in k^n\), the coordinates of \(\phi(x)\) are \(f_1(x), \ldots, f_m(x)\).

Let \(X, Y\) be affine algebraic varieties in \(k^n, k^m\) respectively.
A mapping \(\phi : X \to Y\) is said to be \emph{regular} if \(\phi\) is the restriction to \(X\) of a polynomial mapping from \(k^n\) to \(k^m\).

If \(\eta\) is a polynomial function on \(Y\), then \(\eta \circ \phi\) is a polynomial function on \(X\).
Hence \(\phi\) induces a \(k\)-algebra homomorphism \(P(Y) \to P(X)\), namely \(\eta \mapsto \eta \circ \phi\).
Show that in this way we obtain a one-to-one correspondence between the regular mappings \(X \to Y\) and the \(k\)-algebra homomorphisms \(P(Y) \to P(X)\).
\end{exercise}