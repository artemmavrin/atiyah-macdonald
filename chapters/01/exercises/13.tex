\section{Construction of an algebraic closure of a field \textup{(E. Artin)}}

\begin{exercise}
Let \(K\) be a field and let \(\Sigma\) be the set of all irreducible monic polynomials \(f\) in one indeterminate with coefficients in \(K\).
Let \(A\) be the polynomial ring over \(K\) generated by indeterminates \(x_f\), one for each \(f \in \Sigma\).
Let \(\mathfrak a\) be the ideal of \(A\) generated by the polynomials  \(f(x_f)\) for all \(f\in \Sigma\).
Show that \(\mathfrak a \neq (1)\).

Let \(\mathfrak m\) be a maximal ideal of \(A\) containing \(\mathfrak a\), and let \(K_1 = A/\mathfrak m\).
Then \(K_1\) is an extension field of \(K\) in which each \(f \in \Sigma\) has a root.
Repeat the construction with \(K_1\) in place of \(K\), obtaining a field \(K_2\), and so on.
Let \(L = \bigcup_{n=1}^\infty K_n\).
Then \(L\) is a field in which each \(f \in \Sigma\) splits completely into linear factors.
Let \(\bar K\) be the set of all elements of \(L\) which are algebraic over \(K\).
Then \(\bar K\) is an algebraic closure of \(K\).
\end{exercise}

\begin{solution}
Suppose \(\mathfrak{a} = (1)\).
Then there is a finite set \(F \subseteq \Sigma\) and a family \((g_f)_{f \in F}\) of elements of \(A\) such that \(\sum_{f \in F} g_f f(x_f) = 1\).
Let \(G\) be the finite set of all \(f \in \Sigma\) such that either \(f \in F\) or \(x_f\) occurs in a non-zero term of \(g_h\) for some \(h \in F\).
Let \(g_f = 0\) for \(f \in G \setminus F\).
Then
\begin{equation}
\label{eq:1.13:polynomial equation}
\sum_{f \in G} g_f f(x_f) = 1,
\end{equation}
and there are only finitely many indeterminates occurring in \eqref{eq:1.13:polynomial equation}: they are among the \(x_f\), for \(f \in G\).

Therefore, we may restate \eqref{eq:1.13:polynomial equation} as an equation in \(K[x_1, \ldots, x_n]\) for some positive integer \(n\): there are polynomials \(g_1, \ldots, g_n \in K[x_1, \ldots, x_n]\) and \(f_1, \ldots, f_n \in \Sigma\) such that
\begin{equation}
\label{eq:1.13:polynomial equation 2}
g_1 f_1(x_1) + \cdots + g_n f_n(x_n) = 1.
\end{equation}
Moreover, we may choose \(n\) to be the smallest positive integer such that such an equation holds.

Define \(B = K[x_1, \ldots, x_{n - 1}]\), and let \(\mathfrak{b}\) be the ideal of \(B\) generated by \(f_1(x_1), \ldots, f_{n - 1}(x_{n - 1})\) (if \(n = 1\), take \(B = K\) and \(\mathfrak{b} = (0)\)).
The minimality of \(n\) implies that \(\mathfrak{b} \neq (1)\).
Consequently, by Corollary~1.4, there is a maximal ideal \(\mathfrak{m}\) of \(B\) such that \(\mathfrak{b} \subseteq \mathfrak{m}\).
Let \(f_n^*(x_n)\) be the image of \(f_n(x_n)\) in \((B / \mathfrak{m})[x_n]\).
By \eqref{eq:1.13:polynomial equation 2}, \(f_n^*(x_n)\) is a unit.
However, \(f_n^*(x_n)\) is a non-constant, monic polynomial, and \(B / \mathfrak{m}\) is a field, so \(f_n^*(x_n)\) cannot be a unit by \expartref{1}{2}{i}.
Thus, we have arrived at a contradiction, whence \(\mathfrak{a} \neq (1)\).

Next, using Corollary~1.4, choose a maximal ideal \(\mathfrak{m}\) of \(A\) such that \(\mathfrak{a} \subseteq \mathfrak{m}\), and let \(K_1 = A / \mathfrak{m}\).
Then \(K_1\) is an extension field of \(K\) via the composition \(K \to A \to A / \mathfrak{m}\).
Moreover, if \(f \in \Sigma\) and \(\alpha_f\) denotes the image of \(x_f \in A\) in \(K_1\), then \(f(\alpha_f) = 0\) since \(f(x_f) \in \mathfrak{m}\).
Thus, every polynomial in \(\Sigma\) (i.e., every monic, irreducible polynomial with coefficients in \(K\)) has a root in \(K_1\).

Repeat this construction countably many times to get a sequence of field extensions \(K \subseteq K_1 \subseteq K_2 \subseteq \cdots\), and let \(L\) be the union\footnote{%
Formally, we have a direct system \(K \to K_1 \to K_2 \to K_3 \to \cdots\), and \(L\) is the direct limit (a colimit) \(L = \varinjlim K_n\).
We identify each \(K_n\) with its image in \(L\) under the canonical ring homomorphism \(K_n \to L\).
}
\(L = \bigcup_{n=1}^\infty K_n\).
Then \(L\) is a field.
Let \(\bar{K}\) be the subfield of \(L\) consisting of all elements of \(L\) which are algebraic over \(K\).
Now suppose \(f\) is a monic, irreducible polynomial with coefficients in \(\bar{K}\).
There is some positive integer \(n\) such that all the coefficients of \(f\) belong to \(K_n\), so \(f\) has a root \(\alpha \in K_{n+1}\).
Since \(\bar{K}(\alpha) / \bar{K}\) is a finite extension and \(\bar{K} / K\) is algebraic, it follows that \(\alpha\) is algebraic over \(K\), and so \(\alpha \in \bar{K}\).
Thus, every monic irreducible polynomial over \(\bar{K}\) has a root in \(\bar{K}\), so \(\bar{K}\) is algebraically closed.
Since \(\bar{K}\) is also algebraic over \(K\), it follows that \(\bar{K}\) is an algebraic closure of \(K\).\footnote{%
This construction is carried out in, e.g., \cite[Chapter V, Theorem 2.5 and Corollary 2.6]{LangAlgebra}.
The construction is simplified in \cite{GilmerAlgebraicClosure} by showing that \(K_1\) itself contains an algebraic closure of \(K\), so constructing \(K_2, K_3, \ldots\) is redundant.
}%
\end{solution}