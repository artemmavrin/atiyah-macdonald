\section{Construction of an algebraic closure of a field (E. Artin)}

\begin{exercise}
Let \(K\) be a field and let \(\Sigma\) be the set of all irreducible monic polynomials \(f\) in one indeterminate with coefficients in \(K\).
Let \(A\) be the polynomial ring over \(K\) generated by indeterminates \(x_f\), one for each \(f \in \Sigma\).
Let \(\mathfrak a\) be the ideal of \(A\) generated by the polynomials  \(f(x_f)\) for all \(f\in \Sigma\).
Show that \(\mathfrak a \neq (1)\).

Let \(\mathfrak m\) be a maximal ideal of \(A\) containing \(\mathfrak a\), and let \(K_1 = A/\mathfrak m\).
Then \(K_1\) is an extension field of \(K\) in which each \(f \in \Sigma\) has a root.
Repeat the construction with \(K_1\) in place of \(K\), obtaining a field \(K_2\), and so on.
Let \(L = \bigcup_{n=1}^\infty K_n\).
Then \(L\) is a field in which each \(f \in \Sigma\) splits completely into linear factors.
Let \(\bar K\) be the set of all elements of \(L\) which are algebraic over \(K\).
Then \(\bar K\) is an algebraic closure of \(K\).
\end{exercise}

\begin{solution}
Suppose \(\mathfrak{a} = (1)\).
Then there is a finite set \(F \subseteq \Sigma\) and a family \((g_f)_{f \in F}\) of elements of \(A\) such that \(1 = \sum_{f \in F} g_f f(x_f)\).
Let \(G\) be the finite set of all \(f \in \Sigma\) such that either \(f \in F\) or \(x_f\) occurs in a non-zero term of \(g_h\) for some \(h \in F\).
Let \(g_f = 0\) for \(f \in G \setminus F\).
Then
\begin{equation}
\label{eq:1.13:polynomial equation}
1 = \sum_{f \in G} g_f f(x_f),
\end{equation}
and there are only finitely many indeterminates occurring in \eqref{eq:1.13:polynomial equation}: they are among the \(x_f\), for \(f \in G\).

Therefore, we may restate \eqref{eq:1.13:polynomial equation} as an equation in \(K[x_1, \ldots, x_n]\) for some positive integer \(n\): there are polynomials \(g_1, \ldots, g_n \in K[x_1, \ldots, x_n]\) and \(f_1, \ldots, f_n \in \Sigma\) such that \(1 = \sum_{i=1}^n g_i f_i(x_i)\).
Moreover, we may choose \(n\) to be the smallest positive integer such that such an equation holds.
It must be the case that \(n > 1\), or else \(f_1 \in \Sigma\) would be a unit in \(K[x]\), but the only units in \(K[x]\) are the nonzero elements of \(K\) by \expartref{1}{2}{i}, and \(f_1\) is non-constant since \(f_1\) is irreducible.

\end{solution}