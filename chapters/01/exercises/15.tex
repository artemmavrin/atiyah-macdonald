\subsection{The prime spectrum of a ring}

\begin{exercise}
Let \(A\) be a ring and let \(X\) be the set of all prime ideals of \(A\).
For each subset \(E\) of \(A\), let \(V(E)\) denote the set of all prime ideals of \(A\) which contain \(E\).
Prove that
\begin{parts}
\part
if \(\mathfrak a\) is the ideal generated by \(E\), then \(V(E) = V(\mathfrak a) = V(r(\mathfrak a))\).
\part
\(V(0) = X\), \(V(1) = \emptyset\).
\part
if \((E_i)_{i\in I}\) is any family of subsets of \(A\), then
\begin{equation*}
V\left(\bigcup_{i\in I} E_i\right)
= \bigcap_{i\in I} V(E_i).
\end{equation*}
\part
\(V(\mathfrak a \cap \mathfrak b) = V(\mathfrak a \mathfrak b) = V(\mathfrak a) \cup V(\mathfrak b)\) for any ideals \(\mathfrak a, \mathfrak b\) of \(A\).
\end{parts}
These results show that the sets \(V(E)\) satisfy the axioms for closed sets in a topological space.
The resulting topology is called the \emph{Zariski topology}.
The topological space \(X\) is called the \emph{prime spectrum} of \(A\), and is written \(\Spec(A)\).
\end{exercise}

\begin{partsolution}{1}{15}{i}
If \(\mathfrak{p}\) is a prime ideal of \(A\), then \(E \subseteq \mathfrak{p}\) if and only if \(\mathfrak{a} \subseteq \mathfrak{p}\), so \(V(E) = V(\mathfrak{a})\).
Moreover, \(\mathfrak{a} \subseteq r(\mathfrak{a})\). so \(V(r(\mathfrak{a})) \subseteq V(\mathfrak{a})\).
Conversely, \(r(\mathfrak{a}) = \bigcap V(\mathfrak{a})\) by Proposition~1.14, so if \(\mathfrak{p}\) is a prime ideal of \(A\) such that \(\mathfrak{a} \subseteq \mathfrak{p}\), then \(r(\mathfrak{a}) \subseteq \mathfrak{p}\) as well.
Consequently, \(V(\mathfrak{a}) \subseteq V(r(\mathfrak{a}))\).
\end{partsolution}

\begin{partsolution}{1}{15}{ii}
\(V(0) = X\) since \(0\) belongs to every prime ideal of \(A\).
\(V(1) = \emptyset\) since prime ideals do not contain units (because they are proper ideals).
\end{partsolution}

\begin{partsolution}{1}{15}{iii}
If \(\mathfrak{p}\) is a prime ideal of \(A\), then \(\bigcup_{i \in I} E_i \subseteq \mathfrak{p}\) if and only if \(E_i \subseteq \mathfrak{p}\) for all \(i \in I\), so \(V\left(\bigcup_{i\in I} E_i\right) = \bigcap_{i\in I} V(E_i)\).
\end{partsolution}

\begin{partsolution}{1}{15}{iv}

\end{partsolution}