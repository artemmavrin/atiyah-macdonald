\begin{exercise}
Let \(\phi : A \to B\) be a ring homomorphism.
Let \(X = \Spec(A)\) and \(Y = \Spec(B)\).
If \(\mathfrak{q} \in Y\), then \(\phi^{-1}(\mathfrak{q})\) is a prime ideal of \(A\), i.e., a point of \(X\).
Hence \(\phi\) induces a mapping \(\phi^* : Y \to X\).
Show that
\begin{parts}
\part
If \(f \in A\) then \({\phi^*}^{-1}(X_f) = Y_{\phi(f)}\), and hence that \(\phi^*\) is continuous.
\part
If \(\mathfrak{a}\) is an ideal of \(A\), then \({\phi^*}^{-1}(V(\mathfrak{a})) = V(\mathfrak{a}^e)\).
\part
If \(\mathfrak{b}\) is an ideal of \(B\), then \(\bar{\phi^*(V(\mathfrak{b}))} = V(\mathfrak{b}^c)\).
\part
If \(\phi\) is surjective, then \(\phi^*\) is a homeomorphism of \(Y\) onto the closed subset \(V(\Ker(\phi))\) of \(X\).
(In particular, \(\Spec(A)\) and \(\Spec(A / \mathfrak{N})\) (where \(\mathfrak{N}\) is the nilradical of \(A\)) are naturally homeomorphic.)
\part
If \(\phi\) is injective, then \(\phi^*(Y)\) is dense in \(X\).
More precisely, \(\phi^*(Y)\) is dense in \(X\) \(\iff\) \(\Ker(\phi) \subseteq \mathfrak{N}\).
\part
Let \(\psi : B \to C\) be another ring homomorphism.
Then \((\psi \circ \phi)^* = \phi^* \circ \psi^*\).
\part
Let \(A\) be an integral domain with just one non-zero prime ideal \(\mathfrak{p}\), and let \(K\) be the field of fractions of \(A\).
Let \(B = (A / \mathfrak{p}) \times K\).
Define \(\phi : A \to B\) by \(\phi(x) = (\bar{x}, x)\), where \(\bar{x}\) is the image of \(x\) in \(A / \mathfrak{p}\).
Show that \(\phi^*\) is bijective but not a homeomorphism.
\end{parts}
\end{exercise}

\begin{partsolution}{i}
If \(f \in A\), then \(\mathfrak{q} \in {\phi^*}^{-1}(X_f)\) if and only if \(\phi^*(\mathfrak{q}) = \phi^{-1}(\mathfrak{q}) \in X_f\), which is the case if and only if \(f \notin \phi^{-1}(\mathfrak{q})\), which occurs if and only if \(\mathfrak{q} \in Y_{\phi(f)}\).
Thus, \({\phi^*}^{-1}(X_f) = Y_{\phi(f)}\).
Consequently, since preimages under \(\phi^*\) of basic open sets in \(X\) are open in \(Y\), it follows that \(\phi^*\) is continuous.
\end{partsolution}

\begin{partsolution}{ii}

\end{partsolution}

\begin{partsolution}{iii}

\end{partsolution}

\begin{partsolution}{iv}

\end{partsolution}

\begin{partsolution}{v}

\end{partsolution}

\begin{partsolution}{vi}

\end{partsolution}

\begin{partsolution}{vii}

\end{partsolution}