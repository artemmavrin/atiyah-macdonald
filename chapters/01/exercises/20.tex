\begin{exercise}
Let \(X\) be a topological space.
\begin{parts}
\part
If \(Y\) is an irreducible (\exref{1}{19}) subspace of \(X\), then the closure \(\overline Y\) of \(Y\) in \(X\) is irreducible.
\part
Every irreducible subspace of \(X\) is contained in a maximal irreducible subspace.
\part
The maximal irreducible subspaces of \(X\) are closed and cover \(X\).
They are called the \emph{irreducible components} of \(X\).
What are the irreducible components of a Hausdorff space?
\part
If \(A\) is a ring and \(X = \Spec(A)\), then the irreducible components of \(X\) are the closed sets \(V(\mathfrak p)\), where \(\mathfrak p\) is a minimal prime ideal of \(A\) (\exref{1}{8}).
\end{parts}
\end{exercise}

\begin{partsolution}{i}
Let \(Y\) be an irreducible subspace of \(X\), and let \(U, V\) be nonempty open subsets of \(\bar{Y}\).
Also, choose open subsets \(U^\prime, V^\prime\) of \(X\) such that \(U = \bar{Y} \cap U^\prime\) and \(V = \bar{Y} \cap V^\prime\).
In particular, \(\bar{Y} \cap U^\prime\) is nonempty, so \(Y \cap U^\prime\) is nonempty.
Similarly, \(Y \cap V^\prime\) is nonempty.
Since \(Y \cap U^\prime\) and \(Y \cap V^\prime\) are nonempty open subsets of the irreducible space \(Y\), their intersection is nonempty.
It follows that \(U \cap V\) is nonempty, so \(\bar{Y}\) is irreducible.
\end{partsolution}

\begin{partsolution}{ii}
Let \(Y\) be an irreducible subspace of \(X\), and let \(\Sigma\) be the set of irreducible subspaces of \(X\) which contain \(Y\).
Let \(C\) be a subset of \(\Sigma\) which is totally ordered by \(\subseteq\), and let \(Z \ = \bigcup C\).
In particular \(Y \subseteq Z\).
Moreover, suppose \(U, V\) are non-empty open subsets of \(Z\).
Choose open subsets \(U^\prime, V^\prime\) of \(X\) such that \(U = Z \cap U^\prime\) and \(V = Z \cap V^\prime\).
Pick points \(u \in U\) and \(v \in V\).
Then \(u, v \in W\) for some \(W \in C\), and so \(W \cap U^\prime\) and \(W \cap V^\prime\) are non-empty open subsets of the irreducible space \(W\), so their intersection is non-empty.
Consequently, \(U \cap V\) is nonempty, so \(Z\) is irreducible.
Thus, \(Z \in \Sigma\) is an upper bound for \(C\).
By Zorn's lemma, it follows that \(\Sigma\) has maximal elements.
\end{partsolution}

\begin{partsolution}{iii}
If \(Y\) is a maximal irreducible subspace of \(X\), then \(\bar{Y}\) is irreducible by \partref{1}{20}{i}, so \(Y = \bar{Y}\) by the maximality of \(Y\), whence \(Y\) is closed.
Moreover, if \(x \in X\), then the singleton \(\{x\}\) is an irreducible subspace of \(X\), and so it is contained in a maximal irreducible subspace by \partref{1}{20}{ii}.
Thus, maximal irreducible subspaces of \(X\) cover \(X\).

In a Hausdorff space, any subspace with at least two distinct points has two open subsets which don't intersect (take non-intersecting open neighborhoods around two of the distinct points, which exist by the Hausdorff condition).
Therefore, the irreducible components of a Hausdorff space are the singleton subspaces.
\end{partsolution}

\begin{partsolution}{iv}
We begin with the following general proposition.

\begin{claim}
\label{claim:closed irreducible subspaces of Spec(A)}
A subset \(Y\) of \(X = \Spec(A)\) is closed and irreducible if and only if \(Y = V(\mathfrak{p})\) for some prime ideal \(\mathfrak{p}\) of \(A\).
\end{claim}

\begin{proof}
(\(\Rightarrow\))
Suppose \(Y\) is a closed, irreducible subset of \(X\).
Since \(Y\) is closed, there is a set \(E \subseteq A\) such that \(Y = V(E)\).
By \expartref{1}{15}{i} and Proposition~1.13, \(Y = V(\mathfrak{p})\) where \(\mathfrak{p}\) is the radical of the ideal of \(A\) generated by \(E\).
We claim that \(\mathfrak{p}\) is prime.

Suppose \(f, g \in A \setminus \mathfrak{p}\).
By Proposition~1.14, \(X_f \cap Y\) and \(X_g \cap Y\) are non-empty.
Thus, \(X_f \cap Y\) and \(X_g \cap Y\) are non-empty open subsets of \(Y\), so by the irreducibility of \(Y\), the intersection
\begin{equation*}
(X_f \cap Y) \cap (X_g \cap Y)
= (X_f \cap X_g) \cap Y
= X_{f g} \cap Y
\end{equation*}
is nonempty (\(X_f \cap X_g = X_{f g}\) by \expartref{1}{17}{i}).
Consequently, there is a prime ideal of \(A\) which contains \(\mathfrak{p}\) but not \(f g\).
By Proposition~1.14 again, \(f, g \notin \mathfrak{p}\), so \(\mathfrak{p}\) is a prime ideal.

(\(\Leftarrow\))
Suppose \(\mathfrak{p}\) is a prime ideal of \(A\), and let \(Y = V(\mathfrak{p})\).
Let \(U, V\) be non-empty open subsets of \(Y\).
Since \(U\) and \(V\) are nonempty, choose \(f, g \in A\) such that \(\emptyset \neq X_f \cap Y \subseteq U\) and \(\emptyset \neq X_g \cap Y \subseteq V\).
Therefore, \(f, g \notin \mathfrak{p}\), so \(f g \notin \mathfrak{p}\).
It follows that \(X_{f g} \cap Y\) is nonempty, and \(X_f \cap X_g = X_{f g}\) by \expartref{1}{17}{i}, so \(U \cap V\) is nonempty.
Therefore, \(Y\) is irreducible.
\end{proof}

Returning to the proof of \partref{1}{20}{iv}, note that \autoref{claim:closed irreducible subspaces of Spec(A)} implies that the assignment \(\mathfrak{p} \mapsto V(\mathfrak{p})\) is an inclusion-reversing bijection between \(\Spec(A)\) and the set of closed, irreducible subspaces of \(\Spec(A)\).
In particular, irreducible components correspond to minimal prime ideals.
\end{partsolution}