\begin{exercise}
Let \(X\) be a topological space.
\begin{parts}
\part
If \(Y\) is an irreducible (\exref{1}{19}) subspace of \(X\), then the closure \(\overline Y\) of \(Y\) in \(X\) is irreducible.
\part
Every irreducible subspace of \(X\) is contained in a maximal irreducible subspace.
\part
The maximal irreducible subspaces of \(X\) are closed and cover \(X\).
They are called the \emph{irreducible components} of \(X\).
What are the irreducible components of a Hausdorff space?
\part
If \(A\) is a ring and \(X = \Spec(A)\), then the irreducible components of \(X\) are the closed sets \(V(\mathfrak p)\), where \(\mathfrak p\) is a minimal prime ideal of \(A\) (\exref{1}{8}).
\end{parts}
\end{exercise}

\begin{partsolution}{i}
Let \(Y\) be an irreducible subspace of \(X\), and let \(U, V\) be nonempty open subsets of \(\bar{Y}\).
Also, choose open subsets \(U^\prime, V^\prime\) of \(X\) such that \(U = \bar{Y} \cap U^\prime\) and \(V = \bar{Y} \cap V^\prime\).
In particular, \(\bar{Y} \cap U^\prime\) is nonempty, so \(Y \cap U^\prime\) is nonempty.
Similarly, \(Y \cap V^\prime\) is nonempty.
Since \(Y \cap U^\prime\) and \(Y \cap V^\prime\) are nonempty open subsets of the irreducible space \(Y\), their intersection is nonempty.
It follows that \(U \cap V\) is nonempty, so \(\bar{Y}\) is irreducible.
\end{partsolution}

\begin{partsolution}{ii}
Let \(Y\) be an irreducible subspace of \(X\), and let \(\Sigma\) be the set of irreducible subspaces of \(X\) which contain \(Y\).
Let \(C\) be a subset of \(\Sigma\) which is totally ordered by \(\subseteq\), and let \(Z \ = \bigcup C\).
In particular \(Y \subseteq Z\).
Moreover, suppose \(U, V\) are non-empty open subsets of \(Z\).
Choose open subsets \(U^\prime, V^\prime\) of \(X\) such that \(U = Z \cap U^\prime\) and \(V = Z \cap V^\prime\).
Pick points \(u \in U\) and \(v \in V\).
Then \(u, v \in W\) for some \(W \in C\), and so \(W \cap U^\prime\) and \(W \cap V^\prime\) are non-empty open subsets of the irreducible space \(W\), so their intersection is non-empty.
Consequently, \(U \cap V\) is nonempty, so \(Z\) is irreducible.
Thus, \(Z \in \Sigma\) is an upper bound for \(C\).
By Zorn's lemma, it follows that \(\Sigma\) has maximal elements.
\end{partsolution}

\begin{partsolution}{iii}
If \(Y\) is a maximal irreducible subspace of \(X\), then \(\bar{Y}\) is irreducible by \partref{1}{20}{i}, so \(Y = \bar{Y}\) by the maximality of \(Y\), whence \(Y\) is closed.
Moreover, if \(x \in X\), then the singleton \(\{x\}\) is an irreducible subspace of \(X\), and so it is contained in a maximal irreducible subspace by \partref{1}{20}{ii}.
Thus, maximal irreducible subspaces of \(X\) cover \(X\).

In a Hausdorff space, any subspace with at least two distinct points has two open subsets which don't intersect (take non-intersecting open neighborhoods around two of the distinct points, which exist by the Hausdorff condition).
Therefore, the irreducible components of a Hausdorff space are the singleton subspaces.
\end{partsolution}

\begin{partsolution}{iv}

\end{partsolution}