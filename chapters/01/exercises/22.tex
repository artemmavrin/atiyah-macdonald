\begin{exercise}
Let \(A = \prod_{i=1}^n A_i\) be the direct product of rings \(A_i\).
Show that \(\Spec(A)\) is the disjoint union of open (and closed) subspaces \(X_i\), where \(X_i\) is canonically homeomorphic with \(\Spec(A_i)\).

Conversely, let \(A\) be any ring.
Show that the following statements are equivalent:
\begin{parts}
\part
\(X = \Spec(A)\) is disconnected.
\part
\(A \cong A_1 \times A_2\) where neither of the rings \(A_1, A_2\) is the zero ring.
\part
\(A\) contains an idempotent \(\neq 0, 1\).
\end{parts}

In particular, the spectrum of a local ring is always connected (\exref{1}{12}).
\end{exercise}

\begin{solution}
First, we claim that \partref{1}{22}{i} implies \partref{1}{22}{iii}.
Suppose \(X = \Spec(A)\) is disconnected, and write \(X = Y \cup Z\) for some disjoint, nonempty, and closed \(Y, Z \subseteq X\).
Choose ideals \(\mathfrak{a}, \mathfrak{b}\) of \(A\) such that \(Y = V(\mathfrak{a})\) and \(Z = V(\mathfrak{b})\).
Since \(Y\) and \(Z\) are nonempty, both \(\mathfrak{a}\) and \(\mathfrak{b}\) are proper ideals of \(A\).
Moreover, since \(Y \cap Z = \emptyset\), there is no prime ideal of \(A\) which contains both \(\mathfrak{a}\) and \(\mathfrak{b}\), so \(\mathfrak{a}\) and \(\mathfrak{b}\) are coprime.
Therefore there exist \(a \in \mathfrak{a}\) and \(b \in \mathfrak{b}\) such that \(a + b = 1\).
Since \(a b \in \mathfrak{a} \mathfrak{b}\) and \(V(\mathfrak{a} \mathfrak{b}) = Y \cup Z = X\) by \expartref{1}{15}{iv}, Proposition~1.8 implies that \(a b\) is nilpotent, so \(a^n b^n = 0\) for some positive integer \(n\).
Since \(a \in r((a^n))\), \(b \in r((b^n))\), and \(a + b = 1\) it follows that \(r((a^n))\) and \(r((b^n))\) are coprime.
Proposition~1.16 then implies that \((a^n)\) and \((b^n)\) are coprime.
Choose \(e \in (a^n)\) and \(f \in (b^n)\) such that \(e + f = 1\).
Then \(e - e^2 = e (1 - e) = e f \in (a^n b^n) = (0)\), so \(e - e^2 = 0\).
Thus, \(e\) is idempotent.
Moreover, \(e \neq 0\) (since otherwise \(1 = 1 - e = f \in (b^n) \subseteq \mathfrak{b}\)), and \(e \neq 1\) (since otherwise \(1 = e \in (a^n) \subseteq \mathfrak{a}\)).
We conclude that \partref{1}{22}{i} implies \partref{1}{22}{iii}.

Next, we claim that \partref{1}{22}{iii} implies \partref{1}{22}{ii}.
Suppose \(e \in A\) is an idempotent such that \(e \neq 0, 1\).
Let \(f = 1 - e\).
Since \(e + f = 1\), the ideals \((e)\) and \((f)\) are coprime.
In particular, it follows that \((e) \cap (f) = (e f)\).
But \(e f = e (1 - e) = e - e^2 = 0\) since \(e\) is idempotent, so \((e) \cap (f) = (0)\).
Now Proposition~1.10 implies that \(A \cong A / (e) \times A / (f)\).
Moreover, \(A / (e)\) and \(A / (f)\) are nonzero since \(e \neq 0, 1\).
Thus, \partref{1}{22}{iii} implies \partref{1}{22}{ii}.

Finally, the following \autoref{claim:1.22} will imply both that \partref{1}{22}{ii} implies \partref{1}{22}{i} and---by induction---that the first part of the problem statement holds.
\end{solution}

\begin{claim}
\label{claim:1.22}
Suppose \(A = A_1 \times A_2\) for some rings \(A_1, A_2\).
Let \(X = \Spec(A)\), \(X_1 = \Spec(A_1)\) and \(X_2 = \Spec(A_2)\).
Let \(\phi_1 : A \to A_1\) and \(\phi_2 : A \to A_2\) be the projection homomorphisms, and let \(\phi_1^* : X_1 \to X\) and \(\phi_2^* : X_2 \to X\) be the associated continuous maps (cf.~\exref{1}{21}).
Then the diagram
\begin{equation*}
\begin{tikzcd}
X_1 \arrow[r, "\phi_1^*"] & X & \arrow[l, "\phi_2^*"'] X_2
\end{tikzcd}
\end{equation*}
is a disjoint union.
\end{claim}

\begin{proof}
Since \(\phi_1\) and \(\phi_2\) are surjective, \expartref{1}{21}{iv} implies that \(\phi_1^*\) and \(\phi_2^*\) are homeomorphisms onto their images, which are closed.
Without loss of generality, identify their images with \(X_1\) and \(X_2\).
Explicitly, \(X_1\) consists of ideals of the form \(\mathfrak{p} \times A_2\), where \(\mathfrak{p}\) is a prime ideal of \(A_1\), and \(X_2\) consists of ideals of the form \(A_1 \times \mathfrak{q}\), where \(\mathfrak{q}\) is a prime ideal of \(A_2\).
In particular, \(X_1 \cap X_2 = \emptyset\).
Moreover, any prime ideal of \(A\) belongs to either \(X_1\) or \(X_2\), so \(X = X_1 \cup X_2\), and we are done.
\end{proof}