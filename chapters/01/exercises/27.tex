\section{Affine algebraic varieties}

\begin{exercise}
Let \(k\) be an algebraically closed field and let
\begin{equation*}
f_\alpha(t_1, \ldots, t_n) = 0
\end{equation*}
be a set of polynomial equations in \(n\) variables with coefficients in \(k\).
The set \(X\) of all points \(x = (x_l, \ldots, x_n) \in k^n\) which satisfy these equations is an \emph{affine algebraic variety}.

Consider the set of all polynomials \(g \in k[t_l, \ldots, t_n]\) with the property that \(g(x) = 0\) for all \(x \in X\).
This set is an ideal \(I(X)\) in the polynomial ring, and is called the \emph{ideal of the variety \(X\)}.
The quotient ring
\begin{equation*}
P(X) = k[t_1, \ldots, t_n] / I(X)
\end{equation*}
is the ring of polynomial functions on \(X\), because two polynomials \(g, h\) define the same polynomial function on \(X\) if and only if \(g - h\) vanishes at every point of \(X\),
that is, if and only if \(g - h \in I(X)\).

Let \(\xi_i\) be the image of \(t_i\) in \(P(X)\).
The \(\xi_i\) (\(1 \leq i \leq n\)) are the \emph{coordinate functions} on \(X\):
if \(x \in X\), then \(\xi_i(x)\) is the \(i\)th coordinate of \(x\).
\(P(X)\) is generated as a \(k\)-algebra by the coordinate functions, and is called the \emph{coordinate ring} (or affine algebra) of \(X\).

As in \exref{1}{26}, for each \(x \in X\) let \(\mathfrak{m}_x\) be the ideal of all \(f \in P(X)\) such that \(f(x) = 0\);
it is a maximal ideal of \(P(X)\).
Hence, if \(\tilde{X} = \Max(P(X))\), we have defined a mapping \(\mu : X \to \tilde{X}\), namely \(x \mapsto \mathfrak{m}_x\).

It is easy to show that \(\mu\) is injective:
if \(x \neq y\), we must have \(x_i \neq y_i\) for for some \(i\) (\(1 \leq i \leq n\)), and hence \(\xi_i - x_i\) is in \(\mathfrak{m}_x\) but not in \(\mathfrak{m}_y\), so that \(\mathfrak{m}_x \neq \mathfrak{m}_y\).
What is less obvious (but still true) is that \(\mu\) is \emph{surjective}.
This is one form of Hilbert's Nullstellensatz (see \chref{7}).
\end{exercise}