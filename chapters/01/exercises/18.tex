\begin{exercise}
For psychological reasons it is sometimes convenient to denote a prime ideal of \(A\) by a letter such as \(x\) or \(y\) when thinking of it as a point of \(X = \Spec(A)\).
When thinking of \(x\) as a prime ideal of \(A\), we denote it by \(\mathfrak p_x\) (logically, of course, it is the same thing).
Show that
\begin{parts}
\part
the set \(\{x\}\) is closed (we say that \(x\) is a ``closed point'') in \(\Spec(A)\) \(\iff\) \(\mathfrak p_x\) is maximal;
\part
\(\overline{\{x\}} = V(\mathfrak p_x)\);
\part
\(y \in \overline{\{x\}}\) \(\iff\) \(\mathfrak p_x \subseteq \mathfrak p_y\);
\part
\(X\) is a \(T_0\)-space (this means that if \(x,y\) are distinct points of \(X\), then either there is a neighborhood of \(x\) which does not contains \(y\), or else there is a neighborhood of \(y\) which does not contain \(x\)).
\end{parts}
\end{exercise}

\begin{partsolution}{i}
\(\{x\}\) is closed if and only if \(\{x\} = V(\mathfrak{p}_x)\) (by \partref{1}{18}{ii}), and this is the case if and only there is no prime ideal of \(A\) which properly contains \(\mathfrak{p}_x\).
By Corollary~1.4, it follows that \(\{x\}\) is closed if and only if \(\mathfrak{p}_x\) is maximal.
\end{partsolution}

\begin{partsolution}{ii}
Stare at \partref{1}{18}{iii}.
\end{partsolution}

\begin{partsolution}{iii}
(\(\Rightarrow\))
Suppose \(y \in \bar{\{x\}}\).
Then \(y\) belongs to every closed subset of \(X\) to which \(x\) belongs.
In particular, \(V(\mathfrak{p}_x)\) is closed and \(x \in V(\mathfrak{p}_x)\), so \(y \in V(\mathfrak{p}_x)\), whence \(\mathfrak{p}_x \subseteq \mathfrak{p}_y\).

(\(\Leftarrow\))
Suppose \(\mathfrak{p}_x \subseteq \mathfrak{p}_y\).
Let \(C\) be a closed subset of \(X\) with \(x \in C\).
Choose \(E \subseteq A\) such that \(C = V(E)\).
Since \(x \in C\), we have \(E \subseteq \mathfrak{p}_x \subseteq \mathfrak{p}_y\), so \(y \in C\).
Thus, \(y\) belongs to every closed subset of \(X\) to which \(x\) belongs, so \(y \in \bar{\{x\}}\).
\end{partsolution}

\begin{partsolution}{iv}
Suppose \(x, y \in X\) are distinct points.
Then either \(\mathfrak{p}_x \not\subseteq \mathfrak{p}_y\) or \(\mathfrak{p}_y \not\subseteq \mathfrak{p}_x\).
By \partref{1}{18}{iii}, either \(y \in X \setminus \bar{\{x\}}\) or \(x \in X \setminus \bar{\{y\}}\).
In the former case, \(X \setminus \bar{\{x\}}\) is an open neighborhood of \(y\) which does not contain \(x\), and in the latter case, \(X \setminus \bar{\{y\}}\) is an open neighborhood of \(x\) which does not contain \(y\).
Thus, \(X\) is a \(T_0\) space.
\end{partsolution}