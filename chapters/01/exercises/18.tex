\begin{exercise}
For psychological reasons it is sometimes convenient to denote a prime ideal of \(A\) by a letter such as \(x\) or \(y\) when thinking of it as a point of \(X = \Spec(A)\).
When thinking of \(x\) as a prime ideal of \(A\), we denote it by \(\mathfrak p_x\) (logically, of course, it is the same thing).
Show that
\begin{parts}
\part
the set \(\{x\}\) is closed (we say that \(x\) is a ``closed point'') in \(\Spec(A)\) \(\iff\) \(\mathfrak p_x\) is maximal;
\part
\(\overline{\{x\}} = V(\mathfrak p_x)\);
\part
\(y \in \overline{\{x\}}\) \(\iff\) \(\mathfrak p_x \subseteq \mathfrak p_y\);
\part
\(X\) is a \(T_0\)-space (this means that if \(x,y\) are distinct points of \(X\), then either there is a neighborhood of \(x\) which does not contains \(y\), or else there is a neighborhood of \(y\) which does not contain \(x\)).
\end{parts}
\end{exercise}

\begin{partsolution}{i}

\end{partsolution}

\begin{partsolution}{ii}

\end{partsolution}

\begin{partsolution}{iii}

\end{partsolution}

\begin{partsolution}{iv}

\end{partsolution}