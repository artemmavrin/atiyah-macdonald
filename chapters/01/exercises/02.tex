\begin{exercise}
Let \(A\) be a ring and let \(A[x]\) be the ring of polynomials in an indeterminate \(x\), with coefficients in \(A\).
Let \(f = a_0 + a_1 x + \cdots + a_n x^n \in A[x]\).
Prove that
\begin{parts}
\part
\(f\) is a unit in \(A[x]\) \(\iff\) \(a_0\) is a unit in \(A\) and \(a_1,\ldots,a_n\) are nilpotent.
[If \(b_0+b_1 x + \cdots + b_m x^m\) is the inverse of \(f\), prove by induction on \(r\) that \(a_n^{r+1} b_{m-r} = 0\).
Hence show that \(a_n\) is nilpotent, and then use Ex. 1.]
\part
\(f\) is a nilpotent \(\iff\) \(a_0,a_1,\ldots,a_n\) are nilpotent.
\part
\(f\) is a zero-divisor \(\iff\) there exists \(a \neq 0\) in \(A\) such that \(a f = 0\).
[Choose a polynomial \(g = b_0 + b_1 x + \cdots + b_m x^m\) of least degree \(m\) such that \(f g = 0\).
Then \(a_n b_m = 0\), hence \(a_n g = 0\) (because \(a_n g\) annihilates \(f\) and has degree \(< m\)).
Now show by induction that \(a_{n-r} g = 0\) (\(0 \leq r \leq n\)).]
\part
\(f\) is said to be \emph{primitive} if \((a_0,a_1,\ldots,a_n)=(1)\).
Prove that if \(f,g\in A[x]\), then \(f g\) is primitive \(\iff\) \(f\) and \(g\) are primitive.
\end{parts}
\end{exercise}

\begin{partsolution}{1}{2}{i}
(\(\Leftarrow\))
If \(a_1,\ldots,a_n\) are nilpotent in \(A\), then they are also nilpotent in \(A[x]\).
Since the nilradical of \(A[x]\) is an ideal, \(a_1 x + \cdots + a_n x^n\) is nilpotent.
Moreover, if \(a_0\) is a unit, then
\begin{equation*}
f = a_0 + (a_1 x + \cdots + a_n x^n)
\end{equation*}
is a unit by Exercise \ref{ex:1.1}.

(\(\Rightarrow\))
Suppose \(f\) is a unit in \(A[x]\), and let \(g \in A[x]\) be its inverse, with constant term \(b_0\).
The constant term of the product \(f g = 1\) is \(a_0 b_0 = 1\), so \(a_0\) is a unit in \(A\).

Next, let \(\mathfrak p\) be a prime ideal of \(A\), and let \(\overline f\) and \(\overline g\) be the reductions of \(f\) and \(g\), respectively, modulo \(\mathfrak p\).
Since \((A/\mathfrak p)[x]\) is an integral domain, the equality \(\overline f \overline g = 1\) implies
\begin{equation*}
\deg \overline f + \deg \overline g = 0.
\end{equation*}
Therefore \(\overline f\) is a constant polynomial in \((A/\mathfrak p)[x]\).
It follows that \(a_1,\ldots,a_n \in \mathfrak p\), and since \(\mathfrak p\) was arbitrarily chosen, it follows from Proposition~1.8 that \(a_1,\ldots, a_n\) are nilpotent.
\end{partsolution}

\begin{partsolution}{1}{2}{ii}
(\(\Leftarrow\))
If \(a_0,\ldots,a_n\) are nilpotent in \(A\), then they are also nilpotent in \(A[x]\).
Since the nilradical of \(A[x]\) is an ideal (Proposition 1.7), it follows that \(f = a_0 + a_1 x + \cdots + a_n x^n\) is nilpotent in \(A[x]\).

(\(\Rightarrow\))
Suppose \(f\) is nilpotent, and let \(\mathfrak p\) be a prime ideal of \(A\).
The reduction of \(f\) modulo \(\mathfrak p\) is nilpotent in the integral domain \((A/\mathfrak p)[x]\), whence it is zero modulo \(\mathfrak p\).
Therefore \(a_0,a_1,\ldots,a_n \in \mathfrak p\), and the arbitrary choice of \(\mathfrak p\) implies that \(a_0,\ldots,a_n\) are in the nilradical of \(A\) by Proposition 1.8.
\end{partsolution}

\begin{partsolution}{1}{2}{iii}
(\(\Leftarrow\))
If there exists a nonzero \(a \in A\) such that \(a f = 0\), then \(f\) is a zero-divisor by definition.

(\(\Rightarrow\))
(This is apparently due to McCoy---cf.~\cite[Theorem 2]{McCoyDivisorsOfZero}.)
Suppose \(f\) is a zero-divisor, and let \(g \in A[x]\) be a nonzero polynomial of least degree \(m\) such that \(g f = 0\).
Write
\begin{equation*}
g = b_0 + b_1 x + \cdots + b_m x^m.
\end{equation*}
The \((m+n)\)th coefficient of the product \(f g = 0\) is \(a_n b_m = 0\), so \(a_n g\) has degree less than \(m\).
Moreover, since \(g f = 0\), we also have \((a_n g) f = 0\), so by the minimality of \(m\), it follows that \(a_n g = 0\).

Next, suppose \(r\in\{0,\ldots,n-1\}\) satisfies
\begin{equation*}
a_n g
= a_{n-1} g
= \cdots
= a_{n-r} g
= 0.
\end{equation*}
Then we have
\begin{align*}
0
= f g
&= a_0 g + a_1 g x + \cdots + a_n g x^n
\\&= a_0 g + a_1 g x + \cdots + a_{n-r-1} g x^{n-r-1}.
\end{align*}
The highest-degree coefficient of the final sum above is \(a_{n-r-1} b_m = 0\), so \(a_{n-r-1} g\) has degree less than \(m\).
Again we have \((a_{n-r-1} g) f = 0\), whence \(a_{n-r-1} g = 0\) by the minimality of the degree of \(g\).

By induction on \(r\), we conclude that \(a_{n-r} g = 0\) for \(0 \leq r \leq n\).
In particular, \(a_j b_m = 0\) for \(0\leq j \leq n\), so that \(b_m f = 0\).
\end{partsolution}

\begin{partsolution}{1}{2}{iv}
Let
\begin{equation*}
f = a_0 + a_1 x + \cdots + a_n x^n,
\qquad
g = b_0 + b_1 x + \cdots + b_m x^m
\end{equation*}
be polynomials in \(A[x]\), and let
\begin{equation*}
g f = c_0 + c_1 x + \cdots + c_{m+n} x^{m+n}
\end{equation*}
be their product, where
\begin{equation*}
c_k = \sum_{\substack{i+j=k \\ 0\leq i\leq n\\0\leq j\leq m}} a_i b_j.
\end{equation*}
Define the ideals \(\mathfrak a = (a_0,\ldots,a_n)\), \(\mathfrak b = (b_0,\ldots,b_m)\), and \(\mathfrak c = (c_0,\ldots,c_{m+n})\) of \(A\).
Clearly \(\mathfrak c \subseteq \mathfrak a \cap \mathfrak b\).

(\(\Rightarrow\))
Suppose \(f g\) is primitive, so that \(\mathfrak c = (1)\).
Since \(\mathfrak c \subseteq \mathfrak a \cap \mathfrak b\), it follows that \(\mathfrak a = \mathfrak b = (1)\), so \(f\) and \(g\) are primitive.

(\(\Leftarrow\))
Suppose \(f g\) is not primitive, so that \(\mathfrak c \neq (1)\).
Then by Corollary~1.4, there exists a maximal ideal \(\mathfrak m\) of \(A\) with \(\mathfrak c \subseteq \mathfrak m\).
Let \(\overline f\) and \(\overline g\) be the reductions of \(f\) and \(g\) modulo \(\mathfrak m\).
Since \((A/\mathfrak m)[x]\) is an integral domain and \(\mathfrak m\) contains the coefficients of \(f g\), it follows that \(\overline f \overline g = 0\), so either \(\overline f = 0\) or \(\overline g = 0\).
Thus, either the coefficients of \(f\) or the coefficients of \(g\) are contained in \(\mathfrak m\), whence either \(\mathfrak a\) or \(\mathfrak b\) is not the unit ideal.
Therefore either \(f\) is not primitive or \(g\) is not primitive.
\end{partsolution}