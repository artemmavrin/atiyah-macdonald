\begin{exercise}
Let \(A\) be a ring and let \(A[x]\) be the ring of polynomials in an indeterminate \(x\), with coefficients in \(A\).
Let \(f = a_0 + a_1 x + \cdots + a_n x^n \in A[x]\).
Prove that
\begin{parts}
\part
\(f\) is a unit in \(A[x]\) \(\iff\) \(a_0\) is a unit in \(A\) and \(a_1,\ldots,a_n\) are nilpotent.
[If \(b_0+b_1 x + \cdots + b_m x^m\) is the inverse of \(f\), prove by induction on \(r\) that \(a_n^{r+1} b_{m-r} = 0\).
Hence show that \(a_n\) is nilpotent, and then use Ex. 1.]
\part
\(f\) is a nilpotent \(\iff\) \(a_0,a_1,\ldots,a_n\) are nilpotent.
\part
\(f\) is a zero-divisor \(\iff\) there exists \(a \neq 0\) in \(A\) such that \(a f = 0\).
[Choose a polynomial \(g = b_0 + b_1 x + \cdots + b_m x^m\) of least degree \(m\) such that \(f g = 0\).
Then \(a_n b_m = 0\), hence \(a_n g = 0\) (because \(a_n g\) annihilates \(f\) and has degree \(< m\)).
Now show by induction that \(a_{n-r} g = 0\) (\(0 \leq r \leq n\)).]
\part
\(f\) is said to be \emph{primitive} if \((a_0,a_1,\ldots,a_n)=(1)\).
Prove that if \(f,g\in A[x]\), then \(fg\) is primitive \(\iff\) \(f\) and \(g\) are primitive.
\end{parts}
\end{exercise}

\begin{partsolution}{1}{2}{i}
(\(\Leftarrow\))
If \(a_1,\ldots,a_n\) are nilpotent in \(A\), then they are also nilpotent in \(A[x]\).
Since the nilradical of \(A[x]\) is an ideal, \(a_1 x + \cdots + a_n x^n\) is nilpotent.
Moreover, if \(a_0\) is a unit, then
\begin{equation*}
f = a_0 + (a_1 x + \cdots + a_n x^n)
\end{equation*}
is a unit by Exercise \ref{ex:1.1}.

(\(\Rightarrow\))
Suppose \(f\) is a unit in \(A[x]\), and let \(g \in A[x]\) be its inverse, with constant term \(b_0\).
The constant term of the product \(f g = 1\) is \(a_0 b_0 = 1\), so \(a_0\) is a unit in \(A\).

Next, let \(\mathfrak p\) be a prime ideal of \(A\), and let \(\overline f\) and \(\overline g\) be the reductions of \(f\) and \(g\), respectively, modulo \(\mathfrak p\).
Since \((A/\mathfrak p)[x]\) is an integral domain, the equality \(\overline f \overline g = 1\) implies
\begin{equation*}
\deg \overline f + \deg \overline g = 0.
\end{equation*}
Therefore \(\overline f\) is a constant polynomial in \((A/\mathfrak p)[x]\).
It follows that \(a_1,\ldots,a_n \in \mathfrak p\), and since \(\mathfrak p\) was arbitrarily chosen, it follows from Proposition~1.8 that \(a_1,\ldots, a_n\) are nilpotent.
\end{partsolution}