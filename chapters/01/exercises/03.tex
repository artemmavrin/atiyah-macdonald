\begin{exercise}
Generalize the results of \exref{1}{2} to a polynomial ring \(A[x_1,\ldots,x_r]\) in several indeterminates.
\end{exercise}

\begin{solution}
First, we generalize \expartref{1}{2}{ii}.

\begin{claim}
\label{claim:1.3.ii}
A polynomial \(f\in A[x_1,\ldots,x_r]\) is nilpotent if and only if its coefficients are all nilpotent.
\end{claim}

\begin{proof}
(\(\Leftarrow\))
Same as the original proof of this direction of \expartref{1}{2}{ii}.

(\(\Rightarrow\))
We induct on \(r\), with the \(r=1\) case being handled by \expartref{1}{2}{ii}.
Consider a nilpotent \(f \in A[x_1,\ldots,x_r]\), with \(r > 1\), and write \(f\) as a single-variable polynomial (in the indeterminate \(x_r\)) with coefficients in \(A[x_1,\ldots,x_{r-1}]\):
\begin{equation*}
f = f_0 + f_1 x_r + \cdots + f_n x_r^n,
\end{equation*}
with each \(f_i \in A[x_1,\ldots,x_{r-1}]\).
Since \(f\) is nilpotent, Exercise \ref{ex:1.2}\ref{ex:1.2.ii} implies that each \(f_i\) is nilpotent in \(A[x_1,\ldots,x_{r-1}]\).
By induction, we conclude that the coefficients of each \(f_i\) are nilpotent, and therefore the coefficients of \(f\) are nilpotent.
\end{proof}

Next, we generalize \expartref{1}{2}{i}.

\begin{claim}
\label{claim:1.3.i}
A polynomial \(f \in A[x_1, \ldots, x_r]\) is a unit if and only if its constant term is a unit in \(A\) and the rest of the coefficients are nilpotent.
\end{claim}

\begin{proof}
(\(\Leftarrow\))
Same as the original proof of this direction of \expartref{1}{2}{i}.

(\(\Rightarrow\))
We induct on \(r\), with the \(r = 1\) case being handled by \expartref{1}{2}{i}.
Consider an invertible \(f \in A[x_1, \ldots, x_r]\), with \(r > 1\).
Again, write
\begin{equation*}
f = f_0 + f_1 x_r + \cdots + f_n x_r^n,
\end{equation*}
with each \(f_i \in A[x_1, \ldots, x_{r-1}]\).
By \expartref{1}{2}{i}, \(f_0\) is a unit and \(f_1, \ldots, f_n\) are nilpotent.
By Claim~\ref{claim:1.3.ii}, the coefficients of \(f_1, \ldots, f_n\) are nilpotent.
Moreover, by induction the constant term of \(f_0\) is a unit and its other coefficients are nilpotent.
Therefore the constant term of \(f\) is a unit, and the remaining coefficients are nilpotent.
\end{proof}

Now we generalize \expartref{1}{2}{iii}.

\begin{claim}
\label{claim:1.3.iii}
A polynomial \(f\in A[x_1, \ldots, x_r]\) is a zero-divisor if and only if there exists a nonzero \(a \in A\) such that \(a f = 0\).
\end{claim}

\begin{proof}
(\(\Leftarrow\))
Nothing to prove.

(\(\Rightarrow\))
We induct on \(r\), with the \(r = 1\) case being handled by \expartref{1}{2}{iii}.
Suppose \(r > 1\) and \(f \in A[x_1, \ldots, x_r]\) is a zero-divisor.
As before, write
\begin{equation*}
f = f_0 + f_1 x_r + \cdots + f_n x_r^n,
\end{equation*}
with each \(f_i \in A[x_1, \ldots, x_{r-1}]\).
Since \(f\) is a zero-divisor in the polynomial ring \(A[x_1, \ldots, x_{r-1}][x_r]\), \expartref{1}{2}{iii} implies that there exists a nonzero \(g \in A[x_1, \ldots, x_{r-1}]\) such that \(g f = 0\).
In particular, \(g f_i = 0\) for all \(i\).
Let \(d_i\) be the highest power of \(x_{r-1}\) occurring in \(f_i\), and let
\begin{equation*}
d = \max\{d_0, d_1, \ldots, d_n\} + 1.
\end{equation*}
Then, define
\begin{equation*}
h
= f_0 + f_1 x_{r-1}^d + \cdots + f_n x_{r-1}^{n d},
\end{equation*}
a polynomial in \(A[x_1, \ldots, x_{r-1}]\).
Then \(g h = 0\), so by induction, there exists a non-zero \(a \in A\) such that \(a h = 0\).
In particular, \(a\) annihilates all the coefficients of \(h\), and these, by the construction of \(h\), include the coefficients of \(f_0, f_1, \ldots, f_n\).
Thus \(a f_i = 0\) for each \(i\), and hence \(a f = 0\).
\end{proof}

Finally, we generalize \expartref{1}{2}{iv}.

\begin{claim}
\label{claim:1.3.iv}
Suppose \(f, g \in A[x_1, \ldots, x_r]\).
Then \(f g\) is primitive if and only if \(f\) and \(g\) are primitive
\end{claim}

\begin{proof}
The same proof as that of \expartref{1}{2}{iv} works here if we define \(\mathfrak a\), \(\mathfrak b\), and \(\mathfrak c\) to be the ideals generated by the coefficients of \(f\), \(g\), and \(f g\), respectively.
\end{proof}
\let\qed\relax
\end{solution}
