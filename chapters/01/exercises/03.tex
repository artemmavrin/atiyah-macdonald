\begin{exercise}
Generalize the results of \exref{1}{2} to a polynomial ring \(A[x_1,\ldots,x_r]\) in several indeterminates.
\end{exercise}

\begin{claim}[\expartref{1}{2}{ii} Generalization]
\label{claim:1.3.ii}
A polynomial \(f\in A[x_1,\ldots,x_r]\) is nilpotent if and only if its coefficients are all nilpotent.
\end{claim}

\begin{proof}
(\(\Leftarrow\))
Same as the original proof of this direction of \expartref{1}{2}{ii}.

(\(\Rightarrow\))
We induct on \(r\), with the \(r=1\) case being handled by \expartref{1}{2}{ii}.
Consider a nilpotent \(f \in A[x_1,\ldots,x_r]\), with \(r > 1\), and write \(f\) as a single-variable polynomial (in the indeterminate \(x_r\)) with coefficients in \(A[x_1,\ldots,x_{r-1}]\):
\begin{equation*}
f = f_0 + f_1 x_r + \cdots + f_n x_r^n,
\end{equation*}
with each \(f_i \in A[x_1,\ldots,x_{r-1}]\).
Since \(f\) is nilpotent, Exercise \ref{ex:1.2}\ref{ex:1.2.ii} implies that each \(f_i\) is nilpotent in \(A[x_1,\ldots,x_{r-1}]\).
By induction, we conclude that the coefficients of each \(f_i\) are nilpotent, and therefore the coefficients of \(f\) are nilpotent.
\end{proof}