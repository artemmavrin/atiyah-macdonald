\begin{exercise}
For each \(f \in A\), let \(X_f\), denote the complement of \(V(f)\) in \(X = \Spec(A)\).
The sets \(X_f\), are open.
Show that they form a basis of open sets for the Zariski topology, and that
\begin{parts}
\part
\(X_f \cap X_g = X_{f g}\);
\part
\(X_f = \emptyset\) \(\iff\) \(f\) is nilpotent;
\part
\(X_f = X\) \(\iff\) \(f\) is a unit;
\part
\(X_f = X_g\) \(\iff\) \(r((f)) = r((g))\);
\part
\(X\) is quasi-compact (that is, every open covering of \(X\) has a finite sub-covering).
\part
More generally, each \(X_f\) is quasi-compact.
\part
An open subset of \(X\) is quasi-compact if and only if it is a finite union of sets \(X_f\).
\end{parts}
The sets \(X_f\) are called \emph{basic open sets} of \(X = \Spec(A)\).

[To prove \partref{1}{17}{v}, remark that it is enough to consider a covering of \(X\) by basic open sets \(X_{f_i}\), (\(i \in I\)).
Show that the \(f_i\) generate the unit ideal and hence that there is an equation of the form
\begin{equation*}
1 = \sum_{i \in J} g_i f_i
\qquad (g_i \in A)
\end{equation*}
where \(J\) is some \emph{finite} subset of \(I\).
Then the \(X_{f_i}\) (\(i \in J\)) cover \(X\).]
\end{exercise}

Before proving the numbered parts of \exref{1}{17}, we prove the following.

\begin{claim}
\label{claim:basic open sets for zariski topology}
The sets \(X_f\), for \(f \in A\), form a basis of open sets for the Zariski topology on \(X\).
\end{claim}

\begin{proof}
It suffices to express every open subset of \(X\) as a union of sets of the form \(X_f\).
Take an open set \(U \subseteq X\).
Then there is some \(E \subseteq A\) such that \(U = X \setminus V(E)\).
By \expartref{1}{15}{iii}, we have
\begin{equation*}
U
= X \setminus V(E)
= X \setminus \bigcap_{f \in E} V(f)
= \bigcup_{f \in E} V(f).
\qedhere
\end{equation*}
\end{proof}

\begin{partsolution}{i}
By \expartref{1}{15}{iv}, if \(f, g \in A\), then
\begin{equation*}
X_f \cap X_g = X \setminus (V(f) \cup V(g)) = X \setminus V(f g) = X_{f g}.
\qedhere
\end{equation*}
\end{partsolution}

\begin{partsolution}{ii}
If \(f \in A\), then \(X_f = \emptyset\) if and only if \(V(f) = X\), which is the case if and only if \(f\) belongs to the nilradical of \(A\) by Proposition~1.8.
\end{partsolution}

\begin{partsolution}{iii}
If \(f \in A\), then \(X_f = X\) if and only if \(V(f) = \emptyset\), which is the case if and only if \(f\) is a unit by Corollary~1.5.
\end{partsolution}

\begin{partsolution}{iv}
(\(\Leftarrow\))
If \(r((f)) = r((g))\), then \(V(f) = V(r((f))) = V(r((g))) = V(g)\) by \expartref{1}{15}{i}, so \(X_f = X_g\).

(\(\Rightarrow\))
Suppose \(X_f \subseteq X_g\), so that \(V(g) \subseteq V(f)\).
By Proposition~1.14, \(r((f)) = \bigcap V(f)\) and \(r((g)) = \bigcap V(g)\), so \(r((f)) \subseteq r((g))\).
Similarly, if \(X_g \subseteq X_f\), then \(r((g)) \subseteq r((f))\).
Thus, if \(X_f = X_g\), then \(r((f)) = r((g))\).
\end{partsolution}

\begin{partsolution}{v}
This follows from \partref{1}{17}{vi} since \(X = X_1\) by \partref{1}{17}{iii}.
\end{partsolution}

\begin{partsolution}{vi}
Let \(f \in A\), and suppose \((U_i)_{i \in I}\) is an open covering of \(X_f\).
For each \(i \in I\), choose \(E_i \subseteq A\) such that \(U_i = X \setminus V(E_i)\).
Then, as in the proof of \autoref{claim:basic open sets for zariski topology}, we have \(U_i = \bigcup_{g \in E_i} X_g\) for all \(i \in I\).
Letting \(E = \bigcup_{i \in I} E_i\), it follows that \(X_f = \bigcup_{g \in E} X_g\).
Consequently, if we can find finitely many \(g_1, \ldots, g_m \in E\) such that \(X_f = X_{g_1} \cup \cdots \cup X_{g_n}\), then we can choose \(i_1, \ldots, i_n \in I\) such that \(g_k \in E_{i_k}\) for all \(k \in \{1, \ldots, n\}\), whence \(U_{i_1}, \ldots, U_{i_n}\) is a finite sub-covering of \((U_i)_{i \in I}\).
Thus, it suffices to show that every covering of \(X_f\) by basic open sets has a finite sub-covering.

Suppose \((f_i)_{i \in I}\) is a family of elements of \(A\) with \(X_f = \bigcup_{i \in I} X_{f_i}\).
Let \(\mathfrak{a}\) denote the ideal of \(A\) generated by \((f_i)_{i \in I}\).
By \expartref{1}{15}{i} and \expartref{1}{15}{iii}, we have
\begin{equation*}
V(f) = \bigcap_{i \in I} V(f_i) = V\left((f_{i \in I})_{i \in I}\right) = V(\mathfrak{a}).
\end{equation*}
By Proposition~1.14, \(r((f)) = r(\mathfrak{a})\).
In particular, \(f \in r(\mathfrak{a})\), so there is a positive integer \(n\) such that \(f^n \in \mathfrak{a}\).
Consequently, there exists a finite set \(J \subseteq I\) and a finite family \((g_i)_{i \in J}\) of elements of \(A\) such that \(f^n = \sum_{i \in J} g_i f_i\).

Now take a point \(\mathfrak{p} \in X_f\) (so that \(f \notin \mathfrak{p}\)).
Then there exists an \(j \in J\) such that \(f_j \notin \mathfrak{p}\) since, otherwise, \(f^n = \sum_{i \in J} g_i f_i\) would belong to \(\mathfrak{p}\), and hence \(f\) would belong to \(\mathfrak{p}\).
Thus, \(\mathfrak{p} \in X_{f_j}\).
It follows that \(X_f = \bigcup_{i \in J} X_{f_i}\), so every covering of \(X_f\) by basic open sets has a finite sub-covering.
Therefore \(X_f\) is quasi-compact.
\end{partsolution}

\begin{partsolution}{vii}
(\(\Leftarrow\))
Finite unions of open (respectively, compact) sets are open (respectively, compact) in any topological space.
Consequently, a finite union of sets of the form \(X_f\) is open since each \(X_f\) is open, and quasi-compact since each \(X_f\) is quasi-compact by \partref{1}{17}{vi}.

(\(\Rightarrow\))
Suppose \(U\) is an open, compact subset of \(X\).
By openness and \autoref{claim:basic open sets for zariski topology}, \(U\) is a union of basic open sets of the form \(X_f\), so, by compactness, \(U\) is a finite union of such sets.
\end{partsolution}