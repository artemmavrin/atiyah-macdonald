\begin{exercise}
For each \(f \in A\), let \(X_f\), denote the complement of \(V(f)\) in \(X = \Spec(A)\).
The sets \(X_f\), are open.
Show that they form a basis of open sets for the Zariski topology, and that
\begin{parts}
\part
\(X_f \cap X_g = X_{f g}\);
\part
\(X_f = \emptyset\) \(\iff\) \(f\) is nilpotent;
\part
\(X_f = X\) \(\iff\) \(f\) is a unit;
\part
\(X_f = X_g\) \(\iff\) \(r((f)) = r((g))\);
\part
\(X\) is quasi-compact (that is, every open covering of \(X\) has a finite sub-covering).
\part
More generally, each \(X_f\) is quasi-compact.
\part
An open subset of \(X\) is quasi-compact if and only if it is a finite union of sets \(X_f\).
\end{parts}
The sets \(X_f\) are called \emph{basic open sets} of \(X = \Spec(A)\).

[To prove \partref{1}{17}{v}, remark that it is enough to consider a covering of \(X\) by basic open sets \(X_{f_i}\), (\(i \in I\)).
Show that the \(f_i\) generate the unit ideal and hence that there is an equation of the form
\begin{equation*}
1 = \sum_{i \in J} g_i f_i
\qquad (g_i \in A)
\end{equation*}
where \(J\) is some \emph{finite} subset of \(I\).
Then the \(X_{f_i}\) (\(i \in J\)) cover \(X\).]
\end{exercise}

Before proving the numbered parts of \exref{1}{17}, we prove the following.

\begin{claim}
The sets \(X_f\), for \(f \in A\), form a basis of open sets for the Zariski topology on \(X\).
\end{claim}

\begin{proof}
It suffices to express every open subset of \(X\) as a union of sets of the form \(X_f\).
Take an open set \(U \subseteq X\).
Then there is some \(E \subseteq A\) such that \(U = X \setminus V(E)\).
By \expartref{1}{15}{iii}, we have
\begin{equation*}
U
= X \setminus V(E)
= X \setminus \bigcap_{f \in E} V(f)
= \bigcup_{f \in E} V(f).
\qedhere
\end{equation*}
\end{proof}

\begin{partsolution}{i}
By \expartref{1}{15}{iv}, if \(f, g \in A\), then
\begin{equation*}
X_f \cap X_g = X \setminus (V(f) \cup V(g)) = X \setminus V(f g) = X_{f g}.
\qedhere
\end{equation*}
\end{partsolution}

\begin{partsolution}{ii}
If \(f \in A\), then \(X_f = \emptyset\) if and only if \(V(f) = X\), which is the case if and only if \(f\) belongs to the nilradical of \(A\) by Proposition~1.8.
\end{partsolution}

\begin{partsolution}{iii}
If \(f \in A\), then \(X_f = X\) if and only if \(V(f) = \emptyset\), which is the case if and only if \(f\) is a unit by Corollary~1.5.
\end{partsolution}

\begin{partsolution}{iv}

\end{partsolution}

\begin{partsolution}{v}

\end{partsolution}

\begin{partsolution}{vi}

\end{partsolution}

\begin{partsolution}{vii}

\end{partsolution}