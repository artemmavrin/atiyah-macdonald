\begin{exercise}
Let \(A\) be a Boolean ring (\exref{1}{11}), and let \(X = \Spec(A)\).
\begin{parts}
\part
For each \(f \in A\), the set \(X_f\) (\exref{1}{17}) is both open and closed in \(X\).
\part
Let \(f_1,\ldots,f_n \in A\).
Show that \(X_{f_1} \cup \cdots \cup X_{f_n} = X_f\) for some \(f \in A\).
\part
The sets \(X_f\) are the only subsets of \(X\) which are both open and closed.
[Let \(Y \subseteq X\) be both open and closed.
Since \(Y\) is open, it is a union of basic open sets \(X_f\).
Since \(Y\) is closed and \(X\) is quasi-compact (\exref{1}{17}), \(Y\) is quasi-compact.
Hence \(Y\) is a finite union of basic open sets; now use \partref{1}{23}{ii} above.]
\part
\(X\) is a compact Hausdorff space.
\end{parts}
\end{exercise}

\begin{claim}
\label{claim:boolean ring prime ideals ultrafilter property}
Let \(\mathfrak{p}\) be a prime ideal of a Boolean ring \(A\), and let \(f \in A\).
Then exactly one of \(f \in \mathfrak{p}\) and \(1 - f \in \mathfrak{p}\) holds.
\end{claim}

\begin{proof}
Since \(A\) is a Boolean ring, \(f(1 - f) = f - f^2 = 0\), so \(f(1 - f) \in \mathfrak{p}\).
Thus, either \(f \in \mathfrak{p}\) or \(1 - f \in \mathfrak{p}\) (and not both, or else \(1 \in \mathfrak{p}\)).
\end{proof}

\begin{partsolution}{i}
Suppose \(f \in A\).
If \(\mathfrak{p}\) is a prime ideal of \(A\), then \(f \notin \mathfrak{p}\) if and only if \(1 - f \in \mathfrak{p}\) by \autoref{claim:boolean ring prime ideals ultrafilter property}.
Thus, \(X_f = V(1 - f)\), so \(X_f\) is both open and closed.
\end{partsolution}

\begin{partsolution}{ii}
Let \(\mathfrak{a}\) be the ideal of \(A\) generated by \(f_1, \ldots, f_n\).
By \expartref{1}{11}{iii}, there exists an \(f \in A\) such that \(\mathfrak{a} = (f)\).
Consequently,
\begin{equation*}
V(f_1) \cap \cdots \cap V(f_n)
= V(\mathfrak{a})
= V(f)
\end{equation*}
by \exref{1}{15}, so, taking complements, \(X_{f_1} \cup \cdots \cup X_{f_n} = X_f\).
\end{partsolution}

\begin{partsolution}{iii}
Suppose \(Y \subseteq X\) is open and closed.
Since \(X\) is quasi-compact by \expartref{1}{17}{v} and \(Y\) is closed, \(Y\) is quasi-compact.
Since \(Y\) is open and quasi-compact, \(Y = X_{f_1} \cup \cdots \cup X_{f_n}\) for some \(f_1, \ldots, f_n \in A\) by \expartref{1}{17}{vii}.
Thus, \(Y = X_f\) for some \(f \in A\) by \partref{1}{23}{ii}.
Therefore, the sets \(X_f\), for \(f \in A\), are the only subsets of \(X\) which are both open and closed.
\end{partsolution}

\begin{partsolution}{iv}
Since \(X\) is quasi-compact by \expartref{1}{17}{v}, it suffices to prove that \(X\) is Hausdorff.
Take \(\mathfrak{p}, \mathfrak{q} \in X\) such that \(\mathfrak{p} \neq \mathfrak{q}\).
Without loss of generality, there exists \(f \in \mathfrak{p}\) such that \(f \notin \mathfrak{q}\).
Consequently, \(1 - f \notin \mathfrak{p}\) and \(1 - f \in \mathfrak{q}\) by \autoref{claim:boolean ring prime ideals ultrafilter property}.
Then \(X_{1-f}\) and \(X_f\) are open neighborhoods of \(\mathfrak{p}\) and \(\mathfrak{q}\), respectively, and \(X_f \cap X_{1 - f} = \emptyset\) by \autoref{claim:boolean ring prime ideals ultrafilter property}.
Thus, \(X\) is Hausdorff.
\end{partsolution}