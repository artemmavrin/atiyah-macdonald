\begin{exercise}
Let \(A\) be a Boolean ring (\exref{1}{11}), and let \(X = \Spec(A)\).
\begin{parts}
\part
For each \(f \in A\), the set \(X_f\) (\exref{1}{17}) is both open and closed in \(X\).
\part
Let \(f_1,\ldots,f_n \in A\).
Show that \(X_{f_1} \cup \cdots \cup X_{f_n} = X_f\) for some \(f \in A\).
\part
The sets \(X_f\) are the only subsets of \(X\) which are both open and closed.
[Let \(Y \subseteq X\) be both open and closed.
Since \(Y\) is open, it is a union of basic open sets \(X_f\).
Since \(Y\) is closed and \(X\) is quasi-compact (\exref{1}{17}), \(Y\) is quasi-compact.
Hence \(Y\) is a finite union of basic open sets; now use \partref{1}{23}{ii} above.]
\part
\(X\) is a compact Hausdorff space.
\end{parts}
\end{exercise}

\begin{partsolution}{i}
Suppose \(f \in A\).
Then \(f(1 - f) = f - f^2 = 0\) since \(A\) is a Boolean ring, so if \(\mathfrak{p}\) is a prime ideal of \(A\), then \(f(1 - f) \in \mathfrak{p}\), so either \(f \in \mathfrak{p}\) or \(1 - f \in \mathfrak{p}\) (and not both, otherwise \(1 \in \mathfrak{p}\)).
Consequently, \(f \notin \mathfrak{p}\) if and only if \(1 - f \in \mathfrak{p}\).
Thus, \(X_f = V(1 - f)\), so \(X_f\) is both open and closed.
\end{partsolution}

\begin{partsolution}{ii}
Let \(\mathfrak{a}\) be the ideal of \(A\) generated by \(f_1, \ldots, f_n\).
By \expartref{1}{11}{iii}, there exists an \(f \in A\) such that \(\mathfrak{a} = (f)\).
Consequently,
\begin{equation*}
V(f_1) \cap \cdots \cap V(f_n)
= V(\mathfrak{a})
= V(f)
\end{equation*}
by \exref{1}{15}, so, taking complements, \(X_{f_1} \cup \cdots \cup X_{f_n} = X_f\).
\end{partsolution}

\begin{partsolution}{iii}

\end{partsolution}

\begin{partsolution}{iv}

\end{partsolution}