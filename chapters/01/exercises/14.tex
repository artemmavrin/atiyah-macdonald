\begin{exercise}
In a ring \(A\), let \(\Sigma\) be the set of all ideals in which every element is a zero-divisor.
Show that the set \(\Sigma\) has maximal elements and that every maximal element of \(\Sigma\) is a prime ideal.
Hence the set of zero-divisors in \(A\) is a union of prime ideals.
\end{exercise}

\begin{solution}
Without loss of generality assume \(A \neq 0\), so that \(\Sigma\) is nonempty.
Let \(C\) be a subset of \(\Sigma\) which is totally ordered by \(\subseteq\).
If \(\mathfrak{a} = \bigcup C\), then \(\mathfrak{a}\) is an ideal of \(A\), and if \(x \in \mathfrak{a}\), then \(x \in \mathfrak{c}\) for some \(\mathfrak{c} \in C\), so \(x\) is a zero-divisor.
Thus, \(\mathfrak{a}\) consists of zero-divisors, so \(\mathfrak{a} \in \Sigma\).
Consequently, \(\Sigma\) has maximal elements by Zorn's Lemma.

Let \(\mathfrak{p}\) be a maximal element of \(\Sigma\).
Suppose \(x, y \in A \setminus \mathfrak{p}\).
Then \(\mathfrak{p}\) is a proper subset of the ideal \(\mathfrak{p} + (x)\), so by the maximality of \(\mathfrak{p}\) in \(\Sigma\), there is a non-zero-divisor \(u \in \mathfrak{p} + (x)\).
Similarly, there is a non-zero-divisor \(v \in \mathfrak{p} + (y)\).
It follows that \(u v\) is a non-zero-divisor, and
\begin{equation*}
u v \in (\mathfrak{p} + (x)) (\mathfrak{p} + (y)) \subseteq \mathfrak{p} + (x y).
\end{equation*}
Since \(\mathfrak{p} + (x y)\) contains a non-zero-divisor, it properly contains \(\mathfrak{p}\).
In particular, \(x y \notin \mathfrak{p}\), so \(\mathfrak{p}\) is a prime ideal.
Thus, maximal elements of \(\Sigma\) are prime.\footnote{%
This result is also a corollary of a general ``Prime Ideal Principle''~\cite{LamReyesPrimeIdealPrinciple} which states that certain ideals which are maximal with respect to some property are prime.
In particular, see \cite[Corollary~3.2]{LamReyesPrimeIdealPrinciple} for the result of \exref{1}{14}.%
}

Finally, let \(x \in A\) be a zero-divisor.
The arguments above still hold when \(\Sigma\) is replaced by the set \(\Sigma_x\) of ideals of \(A\) which contain \(x\) and in which every element is a zero-divisor.
In fact, \(\Sigma_x\) is nonempty since \((x) \in \Sigma_x\), so \(\Sigma_x\) has maximal elements by Zorn's lemma, and every such maximal element is a prime ideal by the same proof as before.
Thus, every zero-divisor in \(A\) belongs to a prime ideal which consists of only zero-divisors, so the set of zero-divisors in \(A\) is a union of prime ideals.
\end{solution}