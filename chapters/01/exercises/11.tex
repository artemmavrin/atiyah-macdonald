\begin{exercise}
A ring \(A\) is \emph{Boolean} if \(x^2 = x\) for all \(x \in A\).
In a Boolean ring \(A\), show that
\begin{parts}
\part
\(2x = 0\) for all \(x\in A\);
\part
every prime ideal \(\mathfrak p\) is maximal, and \(A/\mathfrak p\) is a field with two elements;
\part
every finitely generated ideal in \(A\) is principal.
\end{parts}
\end{exercise}

\begin{partsolution}{i}
If \(x \in A\), then \(x + 1 = \left(x + 1\right)^2 = x^2 + 2 x + 1\), so \(2 x = 0\).
\end{partsolution}

\begin{claim}
\label{claim:boolean integral domains are F2}
If \(A\) is a Boolean ring and an integral domain, then \(A\) is a field with two elements.
\end{claim}

\begin{proof}
If \(x \in A\), then \(x(x - 1) = x^2 - x = 0\) since \(x^2 = x\), so either \(x = 0\) or \(x = 1\) (and not both) since \(A\) is an integral domain.
Thus \(A\) is a field with two elements.
\end{proof}

\begin{partsolution}{ii}
Let \(\mathfrak{p}\) be a prime ideal of \(A\).
Then \(A / \mathfrak{p}\) is an integral domain which is also Boolean, so \autoref{claim:boolean integral domains are F2} implies that \(A / \mathfrak{p}\) is a field with two elements.
It follows, in particular, that \(\mathfrak{p}\) is a maximal ideal.
\end{partsolution}

\begin{partsolution}{iii}

\end{partsolution}