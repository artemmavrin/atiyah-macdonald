\chapter{Modules}
\label{ch:2}

\begin{exercise}
\label{ex:2.1}
Show that $(\mathbf Z / m\mathbf Z) \otimes_{\mathbf Z} (\mathbf Z/n\mathbf Z) = 0$ if $m,n$ are coprime.
\end{exercise}

\begin{proof}
This follows immediately from the following stronger observation.
\end{proof}

\begin{claim}
Let $\mathfrak a, \mathfrak b$ be ideals of a ring $A$.
Then $(A/\mathfrak a) \tensor_A (A/\mathfrak b) \cong A/(\mathfrak a + \mathfrak b)$.
\end{claim}

\begin{proof}

????

????

????

\end{proof}



\begin{exercise}
\label{ex:2.2}
Let $A$ be a ring, $\mathfrak a$ an ideal, $M$ an $A$-module.
Show that $(A/\mathfrak a) \otimes_A M$ is isomorphic to $M/\mathfrak a M$.

\noindent
[Tensor the exact sequence $0 \to \mathfrak a \to A \to A/\mathfrak a \to 0$ with $M$.]
\end{exercise}

\begin{proof}
The map $(A/\mathfrak a) \times M \to M / \mathfrak a M$ given by $(a + \mathfrak a, x) \mapsto a x + \mathfrak a M$ is clearly $A$-bilinear, so it induces a canonical $A$-module homomorphism $\phi : (A/\mathfrak a)\otimes_A M \to M/\mathfrak a M$.

?????

?????

?????

?????

(Here is another proof, assuming $M$ is a flat $A$-module.
We tensor the canonical short exact sequence $0 \to \mathfrak a \to A \to A/\mathfrak a \to 0$ with $M$ to get the short exact sequence
\begin{equation*}
0
\to \mathfrak a \tensor_A M
\to A \tensor_A M
\to (A/\mathfrak a) \tensor_A M
\to 0.
\end{equation*}
We have $A \tensor_A M \cong A$, so our exact sequence becomes
\begin{equation*}
0
\to \mathfrak a \tensor_A M
\to M
\to (A/\mathfrak a) \tensor_A M
\to 0.
\end{equation*}
Since $\mathfrak a \tensor_A M \to M$ is injective, $\mathfrak a \tensor_A M$ is isomorphic to the image of this homomorphism, which is clearly $\mathfrak a M$.

?????

?????

?????

?????

This argument does not work for general $A$-modules since, if $M$ is not flat, then it is possible that $\mathfrak a M \not\cong \mathfrak a \tensor_A M$.
For example, let $k$ be a field, and consider $A=k[x]/(x^2)$.
Let $\mathfrak a$ be the ideal of $A$ generated by $x$.
Then clearly $\mathfrak a^2 = 0$, but we claim that $\mathfrak a \tensor_A \mathfrak a \neq 0$.





\end{proof}



\begin{exercise}
\label{ex:2.3}

\end{exercise}

\begin{exercise}
\label{ex:2.4}
Let $(M_i)$, ($i \in I$) be any family of $A$-modules, and let $M$ be their direct sum.
Prove that $M$ is flat $\iff$ each $M_i$ is flat.
\end{exercise}

\begin{proof}
($\Rightarrow$)
Suppose $M$ is flat, and fix $i \in I$.
Let $f : N \to N^\prime$ be an injective homomorphism of $A$-modules.



?????

?????

?????





($\Leftarrow$)


\end{proof}


\begin{exercise}
\label{ex:2.5}
Let $A[x]$ be the ring of polynomials in one indeterminate over a ring $A$.
Prove that $A[x]$ is a flat $A$-algebra.
[Use Exercise \ref{ex:2.4}.]
\end{exercise}

\begin{proof}
As an $A$-module, $A[x] = \sum_{n=0}^\infty A x^n$ is isomorphic to the direct sum $\bigoplus_{n=0}^\infty A$.
Since $A$ is a flat $A$-module, Exercise \ref{ex:2.4} implies that $A[x]$ is a flat $A$-module, and hence a flat $A$-algebra.
\end{proof}



\begin{exercise}
\label{ex:2.6}

\end{exercise}

\begin{exercise}
\label{ex:2.7}
Let $\mathfrak p$ be a prime ideal in $A$.
Show that $\mathfrak p[x]$ is a prime ideal in $A[x]$.
If $\mathfrak m$ is a maximal ideal in $A$, is $\mathfrak m[x]$ a maximal ideal in $A[x]$?
\end{exercise}

\begin{proof}
Let $\mathfrak p$ be a prime ideal in $A$, and consider the canonical surjective ring homomorphism $A[x] \to (A/\mathfrak p)[x]$.
Its kernel is clearly $\mathfrak p[x]$, whence
\begin{equation*}
A[x] / \mathfrak p[x] \cong (A / \mathfrak p)[x].
\end{equation*}
Since $(A/\mathfrak p)[x]$ is an integral domain, it follows that $\mathfrak p[x]$ is prime.

Moreover, if $\mathfrak m$ is a maximal ideal of $A$, then $\mathfrak m[x]$ is never maximal: we have $A[x] / \mathfrak m[x] \cong (A/\mathfrak m)[x]$ by the argument above, so $\mathfrak m[x]$ is maximal if and only if $(A/\mathfrak m)[x]$ is a field.
Since a polynomial ring is never a field, it follows that $\mathfrak m[x]$ is not maximal.
\end{proof}



\begin{exercise}
\label{ex:2.8}

\end{exercise}

\begin{exercise}
\label{ex:2.9}
Let $0 \to M^\prime \to M \to M^{\prime\prime} \to 0$ be an exact sequence of $A$-modules.
If $M^\prime$ and $M^{\prime\prime}$ are finitely generated, then so is $M$.
\end{exercise}

\begin{proof}
Let $0 \to M^\prime \overset f \to M \overset g \to M^{\prime\prime} \to 0$ be an exact sequence of $A$-modules, where $M^\prime$ and $M^{\prime\prime}$ are finitely generated.
Let $x_1,\ldots,x_n$ be a generating set for $M^\prime$, and let $y_1,\ldots,y_m\in M$ be such that $g(y_1),\ldots,g(y_m)$ is a generating set for $M^{\prime\prime}$.
Let $x \in M$ be given.
Then there exist $b_1,\ldots,b_m \in A$ such that
\begin{equation*}
\phi(x) = b_1 \phi(y_1) + \cdots + b_m \phi(y_m).
\end{equation*}
Therefore,
\begin{equation*}
x - b_1 y_1 - \cdots - b_m y_m \in \Ker(g) = f(M^\prime).
\end{equation*}
Since $f$ is injective, $f(M^\prime) \cong M^\prime$, so $f(M^\prime)$ is generated by $f(x_1),\ldots,f(x_n)$.
Thus, there exist $a_1,\ldots,a_n \in A$ such that
\begin{equation*}
x - b_1 y_1 - \cdots - b_m y_m = a_1 f(x_1) + \cdots + a_n f(x_n).
\end{equation*}
Solving for $x$, we see that $x$ is an $A$-linear combination of $f(x_1),\ldots,f(x_n)$ and $y_1,\ldots,y_m$, and hence $M$ is finiely generated.
\end{proof}











\begin{exercise}
\label{ex:2.10}

\end{exercise}








\begin{exercise}
\label{ex:2.11}

\end{exercise}








\begin{exercise}
\label{ex:2.12}

\end{exercise}








\begin{exercise}
\label{ex:2.13}

\end{exercise}








\section*{Direct limits}






\begin{exercise}
\label{ex:2.14}

\end{exercise}




\begin{exercise}
\label{ex:2.15}

\end{exercise}





\begin{exercise}
\label{ex:2.16}

\end{exercise}






\begin{exercise}
\label{ex:2.17}

\end{exercise}





\begin{exercise}
\label{ex:2.18}

\end{exercise}






\begin{exercise}
\label{ex:2.19}

\end{exercise}




\section*{Tensor products commute with direct limits}




\begin{exercise}
\label{ex:2.20}

\end{exercise}




\begin{exercise}
\label{ex:2.21}

\end{exercise}




\begin{exercise}
\label{ex:2.22}
Let $(A_i, a_{i,j})$ be a direct system of rings and let $\mathfrak N_i$ be the nilradical of $A_i$.
Show that $\dlim \mathfrak N_i$ is the nilradical of $\dlim A_i$.

If each $A_i$ is an integral domain, then $\dlim A_i$ is an integral domain.
\end{exercise}

\begin{proof}

\end{proof}


\begin{exercise}
\label{ex:2.23}

\end{exercise}



\section*{Flatness and $\Tor$}




\begin{exercise}
\label{ex:2.24}

\end{exercise}




\begin{exercise}
\label{ex:2.25}

\end{exercise}




\begin{exercise}
\label{ex:2.26}

\end{exercise}




\begin{exercise}
\label{ex:2.27}

\end{exercise}




\begin{exercise}
\label{ex:2.28}

\end{exercise}
















